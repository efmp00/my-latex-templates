\usepackage{titlesec}
\usepackage[many]{tcolorbox}

% ==================================================

% If you want to use the regular computer modern font (LaTeX default font), make sure to comment the below packages

% Sans serif font
% \renewcommand{\familydefault}{\sfdefault}
% Times New Roman font
% \usepackage{times}

% ==================================================

% Chapter Sans Serif font
\titleformat{\chapter}
  {\normalfont\sffamily\LARGE\bfseries}
  {\thechapter.}
  {1em}
  {}

\titleformat{name = \chapter, numberless}[hang]
    {\LARGE\bfseries\sffamily}
    {}
    {0pt}
    {}

\titleformat{\section}[hang]
    {\bfseries\large\sffamily}{\textup{\thesection.}}
    {1em}
    {}
    []

\titleformat{\subsection}[hang]
    {\bfseries\normalsize\sffamily}{\textup{\thesubsection.}}
    {1em}
    {}
    []

\titleformat{\subsubsection}[hang]
    {\bfseries\normalsize\sffamily}{\textup{\thesubsection.}}
    {1em}
    {}
    []

% Adjust spacing after the chapter title
% \titlespacing*{<command>}{<left>}{<before-sep>}{<after-sep>}
\titlespacing*{\chapter}{0cm}{-1.0cm}{0.50cm}
\titlespacing*{\section}{0cm}{0.50cm}{0.25cm}

% Restart chapter counter after another book part
\makeatletter
\@addtoreset{chapter}{part}
\makeatother

\renewcommand{\partname}{Parte}

% Indent 
\setlength{\parindent}{0pt}
\setlength{\parskip}{1ex}

% Theorems, lemma, corollary, postulate, definition=
\numberwithin{equation}{section}

% ==================================================

% Colors
\definecolor{ejbgcolor}{RGB}{240, 248, 255} 
\definecolor{chbgcolor}{RGB}{14, 77, 146}
\definecolor{darkblue}{RGB}{79,129,189}
\definecolor{coolblue}{RGB}{24,94,145}
\definecolor{coolgreen}{RGB}{54, 151, 80}
\definecolor{rosered}{RGB}{185, 39, 73}

% ==================================================

% Theorem
\newtcbtheorem[number within = section]{teorema}{Teorema}%
    {enhanced,
    colback = darkblue!5,
    colbacktitle = darkblue!5,
    coltitle = black,
    boxrule = 0pt,
    frame hidden,
    borderline west = {0.5mm}{0.0mm}{coolblue},
    fonttitle = \bfseries\sffamily,
    breakable,
    before skip = 3ex,
    after skip = 3ex
}{teorema}

\newtcbtheorem[number within = section]{proposicion}{Proposición}%
    {enhanced,
    colback = darkblue!5,
    colbacktitle = darkblue!5,
    coltitle = black,
    boxrule = 0pt,
    frame hidden,
    borderline west = {0.5mm}{0.0mm}{coolblue},
    fonttitle = \bfseries\sffamily,
    breakable,
    before skip = 3ex,
    after skip = 3ex
}{proposicion}

\newtcbtheorem[number within = section]{corolario}{Corolario}%
    {enhanced,
    colback = black!1,
    colbacktitle = black!1,
    coltitle = black,
    boxrule = 0pt,
    frame hidden,
    borderline west = {0.5mm}{0.0mm}{black},
    fonttitle = \bfseries\sffamily,
    breakable,
    before skip = 3ex,
    after skip = 3ex
}{corolario}

\newtcbtheorem[number within = section]{lema}{Lema}%
    {enhanced,
    colback = black!1,
    colbacktitle = black!1,
    coltitle = black,
    boxrule = 0pt,
    frame hidden,
    borderline west = {0.5mm}{0.0mm}{black},
    fonttitle = \bfseries\sffamily,
    breakable,
    before skip = 3ex,
    after skip = 3ex
}{lema}

\newtcbtheorem[number within = section]{postulado}{Postulado}%
    {enhanced,
    colback = darkblue!5,
    colbacktitle = darkblue!5,
    coltitle = black,
    boxrule = 0pt,
    frame hidden,
    borderline west = {0.5mm}{0.0mm}{coolblue},
    fonttitle = \bfseries\sffamily,
    breakable,
    before skip = 3ex,
    after skip = 3ex
}{nota}

\newtcbtheorem[number within = section]{definicion}{Definición}%
    {enhanced,
    colback = darkblue!5,
    colbacktitle = darkblue!5,
    coltitle = black,
    boxrule = 0pt,
    frame hidden,
    borderline west = {0.5mm}{0.0mm}{coolblue},
    fonttitle = \bfseries\sffamily,
    breakable,
    before skip = 3ex,
    after skip = 3ex
}{definicion}

\newtcbtheorem[number within = section]{ejemplo}{Ejemplo}%
    {enhanced,
    colback = white,
    colbacktitle = white,
    coltitle = black,
    boxrule = 0pt,
    frame hidden,
    borderline west = {0.5mm}{0.0mm}{rosered},
    fonttitle = \bfseries\sffamily,
    breakable,
    before skip = 3ex,
    after skip = 3ex
}{ejemplo}

\newtcbtheorem[number within = section]{problema}{Problema}%
    {enhanced,
    colback = white,
    colbacktitle = white,
    coltitle = black,
    boxrule = 0pt,
    frame hidden,
    borderline west = {0.5mm}{0.0mm}{black},
    fonttitle = \bfseries\sffamily,
    breakable,
    before skip = 3ex,
    after skip = 3ex
}{problema}

\newtcbtheorem[number within = section]{nota}{Nota}%
    {enhanced,
    colback = white,
    colbacktitle = white,
    coltitle = black,
    boxrule = 0pt,
    frame hidden,
    borderline west = {0.5mm}{0.0mm}{coolgreen},
    fonttitle = \bfseries\sffamily,
    breakable,
    before skip = 3ex,
    after skip = 3ex
}{nota}

\tcbuselibrary{skins, breakable}