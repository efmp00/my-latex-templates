\chapter{Esto es un capítulo}
\lipsum[1-1]

\section{Esto es una sección}
\lipsum[2-2]

\begin{teorema}{Teorema del residuo}{}{}
Sea $D$ un dominio y $\gamma$ un contorno tal que $D \subset \gamma$, si $f$ analítica en y sobre $\gamma$, excepto por un número finito de singularidades $z_{1}$, $z_{2}$, $\ldots$, $z_{n}$, entonces
\[\ointctrclockwise\limits_{\gamma} f(z) \dd{z} = 2 \pi \im \sum_{k = 1}^{m}\mathrm{Res}(f(z), z_{k}),\]

donde
\[\mathrm{Res}[f(z),z_{k}] = \dfrac{1}{(n - 1)!}\lim_{z \to 
 z_{k}}\dv[n - 1]{}{z}\left[(z - z_{k})^n f(z)\right].\]
\end{teorema}

\subsection{Esto es una subsección}
\lipsum[3-3]

\begin{definicion}{Principio de equivalencia fuerte}{}{}
El \emph{principio de equivalencia fuerte} establece los siguientes puntos:
\begin{itemize}
    \item La dinámica interna de un sistema gravitacional no puede ser alterada por campos gravitacionales externos más allá de las fuerzas de marea\footnote{No obstante, existen teorías que violan esto, por ejemplo, \textit{Brans-Dicke}, \textit{MOND}, entre otras.}.

    \item Todas las leyes de la naturaleza son las mismas en un campo gravitacional uniforme y en el sistema de referencia acelerado equivalente.

    \item La constante de gravitación universal $G$ es una constante en el tiempo.
\end{itemize}
\end{definicion}

\begin{ejemplo}{}{}{}
Demostrar que
\[
(\hat{X} \hat{Y})^{\dagger} = \hat{Y}^{\dagger} \hat{X}^{\dagger}.
\]
\end{ejemplo}

\begin{problema}{Función digamma}{}{}
Se define la \textit{función digamma} como
\begin{equation}
    \psi(z) = \dv{}{z}\ln\left[\Gamma(z)\right] =\dfrac{\Gamma'(z)}{\Gamma(z)};
\end{equation}

a su vez, se define el $n$-ésimo \textit{número armónico} como
 \begin{equation}
    H_{n} = \sum_{k = 1}^{n}\dfrac{1}{k}.
\end{equation}

\textit{Encontrar una relación entre la función digamma y los números armónicos}.
\end{problema}

\section{Prueba}
A continuación, se propone la demostración de una identidad relacionada a la traza parcial.

\begin{proposicion}{Identidad de una traza parcial}{}{}
Si $\mathcal{X}$ y $\mathcal{Y}$ son espacios euclídeos complejos, entonces 
\[
\Tr_{\mathcal{Y}} [(\hat{X} \otimes \mathds{1}_{\mathcal{Y}})\hat{\rho}] = \hat{X} \Tr_{\mathcal{Y}} (\hat{\rho})
\]

para dos operadores $\hat{X} \in \mathcal{L}(\mathcal{X})$ y $\hat{\rho} \in \mathcal{L}(\mathcal{X} \otimes \mathcal{Y})$.
\end{proposicion}

Sean $\mathcal{X}$ y $\mathcal{Y}$ dos espacios euclídeos complejos asociados a los alfabetos $\Gamma$ y $\Sigma$, respectivamente. Además, sean $\hat{X}$ y $\hat{\rho}$ dos operadores tales que $\hat{X} \in \mathcal{L}(\mathcal{X})$ y $\hat{\rho} \in \mathcal{L}(\mathcal{X} \otimes \mathcal{Y})$. En particular, $\hat{\rho}$ puede ser escrito como
\[
\hat{\rho} = \sum_{i \in \Gamma} \sum_{j \in \Sigma} c_{ij} \hat{A}_{i} \otimes \hat{B}_{j} \in \mathcal{L}(\mathcal{X}, \mathcal{Y}),
\]

donde $c_{ij} \in \mathbb{C}$. Por lo tanto, véase que
\begin{align*}
    (\hat{X} \otimes \mathds{1}_{\mathcal{Y}}) \hat{\rho}
    &= (\hat{X} \otimes \mathds{1}_{\mathcal{Y}})\left(\sum_{i \in \Gamma} \sum_{j \in \Sigma} c_{ij} \hat{A}_{i} \otimes \hat{B}_{j}\right)\\
    &= \sum_{i \in \Gamma} \sum_{j \in \Sigma} c_{ij}(\hat{X} \hat{A}_{i}) \otimes (\mathds{1}_{\mathcal{Y}} \hat{B}_{j}) = \sum_{i \in \Gamma} \sum_{j \in \Sigma} c_{ij} \hat{X} \hat{A}_{i} \otimes \hat{B}_{j}.
\end{align*}

Ahora, se aplica la traza parcial:
\begin{align*}
    \Tr_{\mathcal{Y}}[(\hat{X} \otimes \mathds{1}_{\mathcal{Y}})\hat{\rho}]
    &= \Tr_{\mathcal{Y}}\left(\sum_{i\in\Gamma} \sum_{j\in\Sigma} c_{ij}\hat{X}\hat{A}_{i} \otimes \hat{B}_{j}\right)\\
    &= \sum_{i\in\Gamma} \sum_{j\in\Sigma} c_{ij}\hat{X}\hat{A}_{i} \Tr(\hat{B}_{j}).
\end{align*}

Se aprecia que es posible sacar $\hat{X}$ de la suma. Esto conlleva que
\begin{align*}
    \Tr_{\mathcal{Y}} [(\hat{X} \otimes \mathds{1}_{\mathcal{Y}})\hat{\rho}]
    &= \hat{X} \sum_{i \in \Gamma} \sum_{j \in \Sigma} c_{ij} \hat{A}_{i} \Tr(\hat{B}_{j})\\
    &= \hat{X} \Tr_{\mathcal{Y}} \left(\sum_{i\in\Gamma} \sum_{j \in \Sigma} c_{ij} \hat{A}_{i} \otimes \hat{B}_{j}\right) = \hat{X} \Tr_{\mathcal{Y}} (\hat{\rho}).
\end{align*}

\lipsum[4-4]